\RequirePackage{luatex85}
\documentclass[lualatex]{exam}
\usepackage{luatexja}
\usepackage{mathtools}
\usepackage{xcolor}
\usepackage{mathrsfs}
\usepackage{pifont}
\usepackage{bm}
\usepackage{tikz}
\usepackage{dsfont}
\usepackage{amsmath}
\usepackage{amsfonts}
\usepackage{amssymb}
\usepackage{amsthm}
\usepackage[all]{xy}
\usepackage{bookmark}
\usepackage{xurl}
\hypersetup{unicode,bookmarksnumbered=true,hidelinks,final} 
\newcommand{\cl}[1]{\overline{ #1}  } 
\newcommand{\Int}[1]{#1 ^{ o } }
\newcommand{\bd}[1]{\operatorname{Bd}{#1}}
\newtheorem{theorem}{Theorem}[section]
\newtheorem{proposition}[theorem]{Proposition}
\newtheorem{corollary}[theorem]{Corollary}
\newtheorem{lemma}[theorem]{Lemma}
\newtheorem{definition}[theorem]{Definition}
\newtheorem{claim}[theorem]{Claim}
\theoremstyle{remark}
\newtheorem{remark}{Remark}
\theoremstyle{plain} 
\newtheorem{example}[theorem]{Example}
\newtheorem{exercise}[theorem]{Exercise}

\newcommand{\jtoday}{\number \year 年 \number \month 月 \number \day 日}

\everymath{\displaystyle}
\begin{document}
\centerline{\Large 数学科の学部生ができた方がいい問題集 }
\rightline{Arnold Robinson}
\rightline{\jtoday}
%\maketitle
% \setcounter{chapter}{-1}
\tableofcontents
この文章は作者が数学の学部教育を受けて, このぐらいの問題だったらみんな全員できた方が
いいよなと思いながら作った問題集である. 大雑把にいうと, この問題集は学部3年までの
授業で習う知識の中で最も基礎的な常識さえ理解していれば解ける問題ばっかりである.
問題は分野ごとに分けられていて, 線形代数学, 位相数学, ヒルベルト空間論, 群論, 可換環論, 多様体論, 
複素函数論, ルベーグ積分論, フーリエ解析という構成で展開される. 前半の問題は
具体的な計算より, 抽象論を重視するものが多く.
後半では, 理論より応用を重視する. 
この問題集で出てくる問題は大きい定理の証明を分割して出題しているので, 
一個一個の問題は
あんまり苦労せずすんなり解けるはず. 
さらに, これらの問題をクリアすることで, 
多くの定理の証明
が自力に完成できたということに近いので, 
自信をつけるにはちょうどいいのではないかと思う. 

この問題集の問題を解くことにあたって選択公理を自由に用いて良い. (選択公理警察は近寄らないでください)
\section{線形代数}
このsectionでは体$k$と体$k$上のベクトル空間 $V$を固定する. ベクトル空間がすべて$k$上の
有限次元なものとする. すべての線形写像が$k$-線形写像とする. 
\begin{questions}

  \question 体と体上のベクトル空間の定義を述べよ.

  \question $\forall v \in V$に対して, $-v = (-1)v$をベクトル空間の公理を用いて示せ.
  
  \question $V$上のベクトルの組 $\left( v_1, v_2, \ldots, v_m \right) $ が一次独立, 
  $\left( w_1, \ldots ,w_n \right) $ が$V$を生成するとする. この時,  $m \leq n$が成立することを
  示せ. これをもってベクトル空間の次元がwell-definedであることを示せ.
  
  \question $f:V \to W$を線形写像とする. $\dim \ker f + \dim \mathrm{Im} f= \dim V$であることを示せ. 
  
  \question $f:V\to W$を線形写像とする. 
   \begin{parts}
    \part $f$が単射 $\iff \ker f = \{0\} $ 
    \part $f$が全射 $\iff \mathrm{Im}f =W$
  \end{parts}
  を示せ. 
  
  \question すべてのベクトル空間が基底を持つことをZornの補題を用いて示せ.
  
  \question $V,W$の基底をそれぞれ $\left( v_1,\ldots,v_n \right), \left( w_1, \ldots ,w_m \right)  $ として定める. 
  線形写像$f:V \to W$を固定する. 行列$A \coloneqq \left( a_{ij} \right)_{\substack{1\leq i \leq m \\  1 \leq j \leq n}} $ を$f(v_j) = \sum_{i=1}^{n} a_{ij} w_i$を満たすように定義する. 
  この時, 線形写像$\varphi : k^n \to k^m, x \mapsto Ax $が定義できて, $\mathrm{rank} A \coloneqq \dim \mathrm{Im}\varphi  $とする. 
  $\dim \mathrm{Im}f = \mathrm{rank}A$を示せ.

  \question ベクトル空間$V$とその部分空間 $W$に対して, 
  $V$上の同値関係 $\sim $を $x \sim y \iff x - y \in W$と定義する. $V  /\mathord{W} \coloneqq V/ \mathord{\sim}$ 
   とする. この時, $V /W$に自然な $k$-ベクトル空間の構造が入ることを示せ. 
   (和とスカラー倍を定義する時に, well-defined性を確認せよ) この時, $V/ \mathord{W} $を $V$の
   $W$に関する商ベクトル空間と呼ぶ.
   更に,  $\dim V/ \mathord{W} = \dim V - \dim W$を示せ.
   
  \question $f:V \to W$を線形写像とする. $\pi :V \to  V/\mathord{\ker f} $ を自然な射影とする. 
   $f= \tilde{f} \circ \pi$を満たす線形同型写像$\tilde{f} : V /\mathord{\ker f} \to \mathrm{Im}f$
   が一意的に存在することを示せ. (これが準同型定理という, やっていることは$V $から$U$への全射準同型
   が存在するなら,  $V$において $\ker$の元をすべてゼロとして扱えば, 一部の構造を忘れた商ベクトル空間が
    $U$と同型になる)
   
  \question $f: V \to V$を線形写像とする. 任意の$n \geq 0$に対して,  $\ker f^n  \subseteq \ker f^{n+1}$
   が成立することを示したうえで, ある$m \geq 0$が存在して, $n\geq m \implies \ker f^n = \ker f^m$
   が成立することを示せ. 

  \question $f: V_1 \to V_2$を線形写像とする. $W_i \subseteq V_i \left( i =1,2 \right) $ を
  部分空間とする. $f(W_1) \subseteq W_2$を満たすならば, 自然な線形写像$\tilde{f} :V_1 /\mathord{W_1} \to V_2 /\mathord{W_2}$ 
  がwell-definedであることを示せ. (ここでいう自然な線形写像は$\tilde{f} \circ \pi_1 = \pi_2 \circ f$ を満たすものである
  この式では自然だと感じられないかもしれないが, 写像の図式を書けば自然であることがわかる. ただし$\pi_i$を$V_i$からの自然な射影とする.)

  \question $f: V\to V$を線形写像とする. 任意の$n \geq 0$に対して, 
  \begin{parts}
    \part $f(\ker f^{n+1}) \subseteq \ker f^n$ を示せ. 
    \part 自然な写像
    $\ker f^{n+1} / \mathord{\ker f^n} \to \ker f^n / \mathord{\ker f^{n-1}}$ がwell-definedで, 
    単射であることを示せ.
  \end{parts}

  \question $V$をベクトル空間とし,  ${0} = W_0 \subseteq W_1 \subseteq \ldots \subseteq W_m = V$を部分空間
  の増大列とする. 
  \begin{parts}
    \part $\sum_{i=1}^{m} \dim (W_i / \mathord{W_{i-1}}) = \dim V$ となることを示せ. 
    \part $d_i \coloneqq \dim (W_i / \mathord W_{i-1})$ とおき, $v_1^{(i)}, \ldots , v_{d_i}^{(i)} \in W_i$ を
    $\{\overline{v_1^{(i)}}, \ldots , \overline{v_{d_i}^{(i)}}\} \subseteq W_i / \mathord{W_{i-1}} $ が基底となる
    ようにとる. (ただし, $\overline{v^{(i)}_{n}}$ が$v^{(i)}_n$の自然な射影での像とする.) 
    この時, 
    \begin{equation*}
      \begin{aligned}
        \{v^{(1)}_1, \ldots , v_{d_1}^{(1)}, \ldots v_1^{(2)}, \ldots , v_{d_2}^{(2)}, \ldots , 
        v_1^{(m)}, \ldots , v_{d_m}^{(m)}\} 
      \end{aligned}
    \end{equation*}
    が$V$の基底になることを示せ. 

  \end{parts}

  \question $A$を $n$次正方行列とする. 
  $\det A \coloneqq \sum_{\sigma \in S_n}^{} (-1)^{\sigma} \prod_{i=1}^{n} a_{i,\sigma (i)}  $ 
  と定義する. (ただし, $(-1)^{\sigma}$ は$\sigma$が奇置換の時に $-1$, 偶置換の時 $1$を取る函数とする.) 
  以下の命題を示せ.
  \begin{parts}
    \part $A'$を $A$の異なる$2$行を入れ替えた行列とする. この時, $\det A' = - \det A$. 
    \part  $A'$を $A$のある行に他の行のスカラー倍を足したものとする. この時, $\det A' = \det A$. 
    \part  $A'$を $A$のある行を$c$倍したものとする.  この時, $\det A' = c \det A$.
    \part  $A^{\top}$ を$A$の転置行列とする. この時,  $\det A^{\top} = \det A$. 
    (これをもって, 上の議論は列に置き換えても成立することがわかる.)
  \end{parts}

  \question $A,B$を $n$次正方行列とする.  $\det (AB) = \det A \cdot \det B$を示せ. 

  \question $A$を$n$次正方行列とする.  $A$の余因子行列
  $\mathrm{adj}A$を $\mathrm{adj}A \cdot A = A \cdot \mathrm{adj}A = \det A \cdot I_n$($I_n$が$n$次単位行列) を満たす
  ものとする. 余因子行列を構成し, $\mathrm{adj}(tI_n -A)$が $t$の$n-1$次以下の
  行列係数多項式であることを示せ. それを用いて, 行列$A$の固有多項式$P$に対して, 
   $P(A) = 0$であることを示せ. 

  \question $V$を$n$次元ベクトル空間とする. $f:V \to V$を線形写像で$\exists m \geq 1, f^m = 0$ を満たすとする. 
  \begin{parts}
    \part $m \leq n$を示せ. 
    \part  $W_k \coloneqq  \ker f^k$と定め, $\{0\}= W_0 \subseteq W_1 \subseteq \ldots \subseteq W_m = V$ 
    を部分空間の列とする. この時, $f$によって自然に誘導される線形写像の列
     \begin{equation}
      \begin{aligned}
        W_1 = W_1 / \mathord{W_0} \xleftarrow{\eta_1} W_2 / \mathord{W_1} \xleftarrow{\eta_2} \ldots \xleftarrow{\eta_{k-1}} 
        W_k / \mathord{W_{k-1}} \xleftarrow{\eta_{k}} W_{k+1} / \mathord{W_k} \xleftarrow{\eta_{k+1}} \ldots
        \xleftarrow{\eta_{m-1}} W_{m}/\mathord{W_{m-1}}
      \end{aligned}
    \end{equation} 
    があって, $\eta_k$がすべて単射である. この $\eta_k$の列を明示的に構成せよ. 
    さらに,  $d_k\coloneqq \dim (W_k / \mathord{W_{k-1}})$ とした時, 
    \begin{equation}
      \begin{aligned}
        d_1 \geq d_2 \geq \ldots \geq d_k \geq \ldots \geq d_m \geq 0, \quad \sum_{k=1}^{m} d_k = n
      \end{aligned}
    \end{equation} 
    を示せ. 
    \part $v_1^{(k)}, \ldots ,v_{d_k}^{(k)} \in W_k$ を$\{\overline{v_1^{(k)}}, \ldots , \overline{v_{d_k}^{(k)}}\} $が
    $W_k / \mathord{W_{k-1}}$ の基底となるように取る. 
    この時,  $\{\overline{f(v_1^{(k)})}, \ldots , \overline{f(v_{d_k}^{(k)})}\} $が
    $W_{k-1} / \mathord{W_{k-2}}$ 上で一次独立になることを示せ. 
    \part $f$の表現行列がジョルダン標準形となる $V$上の基底を構成せよ. (これで冪零行列のジョルダン標準形の理論ができた.)
  \end{parts}

  \question 
  \label{q:triangle-matrix} この問題では体$k$を代数閉体とする. つまり, $k$係数のすべての多項式は
  一次式の積に分解できるとする. $A$を$k$上の$n$次正方行列とする. この時, ある$n$ 次正方行列
  $P$ が存在して, 
  $P^{-1}AP$が対角成分$\lambda_1, \ldots ,\lambda_n$が$A$の固有値となるような上三角行列になること
  を次のようにプロセスで証明しよう. 
  $V\coloneqq k^n$ とし, $f_A: V \ni x \mapsto Ax \in V$を考える. 
  \begin{parts}
    \part 固有値$\lambda_n$に属する固有ベクトル$x_n \neq 0$をとり,  $W\coloneqq \langle x_n \rangle$($x_n$の生成する一次元ベクトル空間)とおく.
    この時, $f(W) \subseteq W$を示せ. 
    \part 自然な写像$\overline{f}_A : V / \mathord{W} \to V / \mathord{W}$ を考えて, 次元に関する帰納法で$f_A$に対して
     $f_A$の表現行列が上三角行列となる基底が存在することを示せ. 
  \end{parts}

  \question $f:V\to V$を代数閉体$k$上の$n$ 次元ベクトル空間上の線形写像とする. 
  $f$の固有多項式 $p_f (t) \coloneqq  (t- \lambda_1)^{r_1}\cdots (t- \lambda_k)^{r_k}$ とする. 
  $r_k$を固有値 $\lambda_k$の重複度とする. $k$上の$n$次正方行列 $A$を一つ取り, 
   $A$の固有値 $\alpha $を一つ固定して, $r$を $\alpha $ の固有値とする. 
   問題\ref{q:triangle-matrix}より, ある$n$次正方行列
   $P$が存在して, 三角化の最初の$r$個対角成分が $\alpha $にできる. $U \coloneqq  \ker (f - \alpha \cdot 1_V)^r$とする. 
   ここで, $1_V $を $V$上の恒等写像とする. 
    \begin{parts}
      \part $P^{-1}(A-\alpha I_n)^rP$を計算して, $\mathrm{rank}(A-\alpha I_n)^r = n-r$を示せ, 
      このことより, $\dim U = r$ を示せ. 
      \part $V$のある部分空間 $W$が存在して,  $V = U \oplus W, f(W) \subseteq W$となる
      ことを示せ. また, $f(U) \subseteq U$を確認せよ. 
      \part $p_{f|_U}(t) = (t-\alpha )^r$ となることを確かめよう. 
      \part 以上をもって, $V = \bigoplus_{i=1}^k \ker (f- \lambda_i)^{r_i}$ であり, 
      このもとで, 代数閉体上の $n$次正方行列が常にジョルダン標準形をもつことを示せ. 
   \end{parts}






\end{questions}

\end{document}
