% LuaLaTeX文書; 文字コードはUTF-8
\documentclass[unicode,12pt]{beamer}% 'unicode'が必要
\usepackage{luatexja, mathtools}% 日本語したい
\usepackage[ipaex]{luatexja-preset}% IPAexフォントしたい
\renewcommand{\kanjifamilydefault}{\gtdefault}% 既定をゴシック体に

% あとは欧文の場合と同じ
\usetheme{Copenhagen}

%题目,作者,学校,日期
\title{集合, 全単射}
\subtitle{\fontsize{9pt}{14pt}\textbf{素朴集合論}}
\author{永明}
\institute{数学部会}
\date{2023}
%目录设置
\begin{document}
\begin{frame}
%\label{frame:}
 \titlepage 
\end{frame}
\begin{frame}
  \frametitle{集合}
  任意の性質$P$に対して, その性質を満たすもの全体
   \begin{equation}
    \begin{aligned}
      S_P \coloneqq \{x  \mid \text{$x$が性質$P$ を満たす}\} 
    \end{aligned}
  \end{equation} 
  が存在し, これを集合と呼ぶ. 

  \begin{itemize}
    \item $x \in S_P \iff \text{$x$が性質 $P$を満たす.}$
    \item $x \not\in S_P \iff \text{$x$が性質 $P$を満たさない. }$
  \end{itemize}
  集合$A,B$に対して, 任意の $a \in A$について, $a \in B$が成立するならば, 
  $A$が $B$の部分集合といい, $A \subseteq B$ と書く. 
\end{frame}

\begin{frame}
%\label{frame:}
  \frametitle{直積と対応}

  集合$A \times B$に対して, それらの直積$A \times B$を
  \begin{equation}
    \begin{aligned}
      A \times B \coloneqq \{(a,b)  \mid a \in A, b \in B\} 
    \end{aligned}
  \end{equation}
  と定義する. 

  集合$A,B$に対して,  $A \times B$の部分集合$R \subseteq A \times B$を集合 $A$から$B$への対応と呼ぶ.  
 

  $R$と任意の$(a,b) \in A \times B$に対して, $(a,b) \in R \iff a R b$と書く. 
\end{frame}

\begin{frame}
%\label{frame:}
  \frametitle{写像}
  集合$A,B$と$A$から$B$への対応 $f$に対して, $(A,B,f)$の組が写像であるとは

  任意の$a \in A$に対して, 唯一の$b \in B$が存在して, $a f b$が成立する. 

  この時,  $(A,B,f)$を $f :A \to B$と書き, $a \in A$に対して, $a f b$が成立する
   $b \in B$を$f(a)$と表す. 

   


\end{frame}

\begin{frame}
  \frametitle{逆対応と逆写像}
 集合$A$から $B$への対応 $R$に対して, $R$の逆対応 $R^{-1}$ を次のように定義する: 
 \begin{equation}
   \begin{aligned}
     R^{-1}\coloneqq \{(b,a) \in B \times A  \mid (a,b) \in R\} .
   \end{aligned}
 \end{equation} 
写像$(A,B,f)$に対して,  $(B,A,f^{-1})$が写像になる時, $f^{-1}$を$f$の逆写像と呼ぶ.
\end{frame}

\begin{frame}
%\label{frame:}
  \frametitle{単射, 全射, 全単射}

 写像 $f:A \to B$が
\begin{itemize}
  \item 
  単射であるとは
  
  任意の$a, a' \in A$に対して, $f(a) = f(a')$ならば $a = a'$が成立する. 

\item 全射であるとは

  任意の$b \in B$に対して, ある$a \in A$が存在して, $f(a) = b$が成立する. 

\item 全単射であるとは

   $f$が単射かつ全射.
\end{itemize}

  
\end{frame}



\end{document}

