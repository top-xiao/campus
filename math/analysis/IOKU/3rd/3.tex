\documentclass{jarticle}
\usepackage{xcolor}
\usepackage{amsfonts}
\usepackage{amssymb}
\usepackage{amsthm}
\begin{document}
\title{第三回レポート問題}
\author{習近平}
\date{$\forall \varepsilon \in \mathbb{R}_{ > 0}, 締切-\varepsilon <提出時刻<締切$}
\maketitle
\section*{Q3.1}
\subsection*{(1.1)}
\begin{proof}
\noindent 数列$(b_{n})_{n\in \mathbb{N}}$の有界性と単調性を以下示す:
\\
\noindent 有界性: \\
\noindent $ \mathbf{A} = \left\{k \in \mathbb{N} \mid \frac{1}{2} \le b_{k} <1 \right\}$とする.\\
$ \mathbf{A} = \mathbb{N}$ であることを以下示す: \\
$ \frac{1}{2} \le b_{0}= \frac{1}{2} < 1 $より$0 \in \mathbf{A}$.\\
$ \mathbf{S}_{n+1} := \left\{j \in \mathbb{N} \mid j < n+1 \right\}$とする. \\
$ \mathbf{S}_{n+1} \subset \mathbf{A} $と仮定し, $n+1 \in \mathbf{A} $を示す.\\
仮定より,$ \frac{1}{2} \le b_{n} < 1 $が成り立ち,漸化式に代入すると,\\
$\frac{1}{2} \le b_{n+1}= \frac{{b_{n}}^2+1}{2}< \frac{1+1}{2} < 1$ が得られる.\\
従って,$n+1 \in \mathbf{A} $となり,数学的帰納法より,$ \mathbf{A} = \mathbb{N}$が得られ,\\
数列$(b_{n})_{n\in \mathbb{N}}$の有界性が示された.\\

\noindent 単調性:\\
$n \in \mathbb{N}$を任意に取る.与えられた漸化式より,\\
$ b_{n+1}-b_{n} = \frac{1}{2} (b_{n}^{2} - 2b_{n} +1) =\frac{1}{2} (b_{n} -1)^{2} >0 $\\
よって,数列$(b_{n})_{n\in \mathbb{N}}$が単調列であることがわかる.\\

\noindent 実数の連続性より,数列$(b_{n})_{n\in \mathbb{N}}$が閉区間$[ \frac{1}{2} , 1]$の一点に収束することが示された.
\qed
\end{proof}
\subsection*{(1.2)}
\begin{proof}
数列$(b_{n})_{n\in \mathbb{N}}$が1に収束することを以下示す:\\
$(1)$の結論のもとで,
$$ 
\lim_{n \to \infty} b_{n} = \alpha , (\alpha \in [\frac{1}{2} , 1])
$$ 

とする.\\
$$
\lim_{n \to \infty} b_{n} = \alpha
$$
と収束列に関する四則演算の性質より,\\
$$
\lim_{n \to \infty} b_{n}^{2} = \alpha^{2}
$$
となる.
また,収束列の部分列は収束列と同じ極限を持つことより,
$$
\lim_{n \to \infty} b_{n+1} = \alpha
$$
が得られる.\\
以上のことから,\\
$$
 \alpha = \lim_{n \to \infty} b_{n+1} = \lim_{n \to \infty} \frac{b_{n}^{2}+1}{2} =\frac{1}{2} (\alpha^{2} +1)
$$
ここで,中学校で習った知識をうまく活用してやると以下のことが分かる:\\
$$
 \alpha =1
$$
従って,
$$
\lim_{n \to \infty }b_{n} = 1
$$
 であることが証明された.\qed
 \end{proof}
\subsection*{(2)}
\noindent 収束しない.\\
\begin{proof}
$$
\mathbf{B} = \left\{ n \in \mathbb{N} \mid b_{n} \ge n-1 \right\}
$$
とする.\\
$b_0 =2 \ge -1$より,$ 0 \in \mathbf{B}$ \\
$ \mathbf{S}_{n+1} := \left\{j \in \mathbb{N} \mid j < n+1 \right\}$とする. \\
$ \mathbf{S}_{n+1} \subset \mathbf{B} $と仮定し, $n+1 \in \mathbf{B} $を以下示す.\\
仮定より,$b_n \ge n-1$が成り立つ.この時,漸化式より:\\
$$
b_{n+1} = \frac{1}{2} ( b_n^2 +1 ) \\
$$
$$
b_{n+1} -n \ge \frac{1}{2} (n^2 -2n +1) = \frac{1}{2}(n-1)^2 \ge 0
$$
従って,$ n+1 \in \mathbf{B} $ よって,$ \mathbf{B} = \mathbb{N}$が成り立つ.\\
以上より,$ \forall  n \in \mathbb{N} ; b_n \ge n-1 $が成り立つ.\\
任意の実数$ \beta \in \mathbb{R} $に対し,アルキメデス性より,$ N-1 > \beta $となるような$ N \in \mathbb{N}$が存在する.\\
$n \ge N$を満たすような任意の$n \in \mathbb{N}$ に対し,\\
$$
b_n \ge b_N \ge N-1 > \beta
$$
が成り立つため.数列$(b_{n})_{n\in \mathbb{N}}$が無限大に発散することが示された.
\qed
\end{proof}
\section*{Q3.2}\footnote{$0 \in \mathbb{N} $から$k=0$の時$a_{2k-1}$が定義されていないため,以下$a_{2k-1}$の代わりに$a_{2k+1}$を使わせていただきます.}
\begin{proof}
$\varepsilon \in \mathbb{R}_{>0}$を任意に取る.\\
$$
\lim_{k \to \infty} a_{2k} = \alpha \, より \\
$$
ある$k_1 \in \mathbb{N} $が存在し,$k_1^{'} \ge k_1$ を満たす任意の$k_1^{'} \in \mathbb{N}$ に対し,以下の式が成り立つ:\\
$$
\mid a_{2k^{'}} - \alpha \mid < \varepsilon \\
$$
$$
次に \lim_{k \to \infty } a_{2k+1} = \alpha より\\
$$
ある$k_{2} \in \mathbb{N} $が存在し,$k_2^{'} \ge k_2$ を満たす任意の$k_2^{'} \in \mathbb{N}$ に対し,以下の式が成り立つ:\\
$$
\mid a_{2k_{2}^{'}+1} - \alpha \mid <\varepsilon
$$
数列$(a_{n})_{n\in \mathbb{N}}$が$\alpha$に収束することを以下示す:\\
$$
N := max\left\{ 2k_1,2k_2+1\right\} + k_1 + k_2
$$
$n \ge N $を満たすような$n \in \mathbb{N}$ を任意に取る.\\
この時,$ k \ge max\left\{ k_1,k_2\right\}$を満たす$k \in \mathbb{N}$が存在し,
$(n=2k) \lor (n =2k+1)$が成り立つ.\\
いずれにせよ,
$$
\mid a_n - \alpha \mid < \varepsilon
$$
が成り立つため,数列$(a_{n})_{n\in \mathbb{N}}$が$\alpha$に収束することが示された.
\qed
\end{proof}
\section*{Q3.7}
\begin{proof}
\footnote{この問題って実数が累乗のところに来ているから,対数や微積を使わない解法が僕の脳内に存在しないので,対数や微積の諸性質を宣言せずに使わせていただきます.}\\
条件から簡単に分かること: \\
$$
\prod_{k=1}^{n} (1 + \frac{\alpha}{k} )>1
$$
$$ 
ここで,a_n = \prod_{k=1}^{n} (1 + \frac{\alpha}{k})とする.b_n = \ln a_nと定める.\\
$$
この時\\
$$
b_n = \sum_{k=1}^{n} \ln (1 + \frac{\alpha}{k})であることが簡単にわかる.(気になる人はcheckせよ!)
$$
$x >0$の時$\ln (1+x) < x$が成り立つことより,\\
\begin{eqnarray}
b_n & < & \sum_{k=1}^{n} \frac{\alpha}{k}  = \alpha \sum_{k=1}^{n} \frac{1}{k} \nonumber \\
	& = & \alpha + \alpha \sum_{k=2}^{n} \frac{1}{k} \nonumber \\
\end{eqnarray}
この時,微積分の手法を使うと:\\
$$
\frac{1}{k} = \int_{k-1}^{k} \frac{1}{k} dx <\int_{k-1}^{k} \frac{1}{x} dx= \ln (k) -\ln (k-1)
$$
従って,\\
$$
\sum_{k=2}^{n} \frac{1}{k}< \ln (n)
$$
が得られ,
\begin{equation}
b_n < \alpha + \alpha \ln n \label{1}
\end{equation} 
が成り立つ.\\
また
\begin{eqnarray}
\ln(1+ \frac{\alpha}{k}) & = & \int_0^{\frac{\alpha}{k}} \frac{1}{1+x}dx >\int_0^{\frac{\alpha}{k}}\frac{1}{1+\frac{\alpha}{k}} dx \nonumber \\
			   & = & \frac{\alpha}{k+\alpha} \nonumber \\
\end{eqnarray}
が成り立つから,
$$
\alpha \sum_{k=1}^{n} \frac{1}{k+\alpha} < b_n
$$
が成り立つ.\\
また,\\
$$
\frac{1}{k+\alpha} = \int_{k}^{k+1} \frac{1}{k+\alpha} dx > \int_{k}^{k+1} \frac{1}{x+\alpha} dx= \ln (k+1+\alpha) -\ln (k+\alpha)
$$
より,
$$
\sum_{k=1}^{n} \frac{1}{k+\alpha} > \ln (n+1+\alpha) - \ln (1+\alpha)> \ln n \, -\ln (1+\alpha)
$$
従って,\\
$$
b_n > \alpha \sum_{k=1}^{n} \frac{1}{k+\alpha} > \alpha \ln (n+1+\alpha) - \alpha \ln (1+\alpha)> \alpha \ln n \, -\alpha
$$
が成り立つ.\\
以上より:\\
$$
-\alpha + \alpha \ln n < b_n < \alpha + \alpha \ln n
$$
が得られる,これを指数の形に直すと次のようになる:\\
$$
e^{-\alpha + \alpha \ln n}< a_n <e^{\alpha + \alpha \ln n}
$$
つまり,
$$
e^{-\alpha }n^{\alpha}< \frac{(\alpha +1)...(\alpha +n)}{n!}<e^{\alpha}n^{\alpha}
$$
が成り立つ.\\
以上の議論より:\\
$$
c_1 =e^{\alpha} \, ;c_2 =e^{-\alpha}
$$
と置けば,すべてがうまく行くよ.
\qed
\end{proof}
\section*{\#雜談}
前回対数に関する質問を答えてくれて,ありがとう.今回は流石に対数を使わなければいけないよね.(笑)早く講義中に対数を定義してくれたらありがたいw.\\
今回はおまけ問題を全部やろうとしていたが,時間がなさそうなので,点数になれる部分しか解いていなかった.
今回のおまけ問題を解く時,定数の決め方に悩んでいた.三角関数に関する不等式$\frac{2x}{\pi} < sinx$を思い出して,もしかして,何らかの指数を定数として評価すればいいかに気づいて,雑な不等式を数時間いじってみたらうまくいった.\\
もう一つだね,\LaTeX で行内数式を使うと$\sum_{k=1}^{n}$,下標が右に出ているのが気持ち悪い.何か対策とかあるの?


\end{document}


