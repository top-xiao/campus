\documentclass{jreport}
\usepackage{xcolor}
\usepackage{amsmath}
\usepackage{amsfonts}
\usepackage{amssymb}
\usepackage{amsthm}
\usepackage{bm}
\usepackage{romannum}
\usepackage[dvipdfmx,hidelinks]{hyperref}
\usepackage{pxjahyper}
\usepackage{framed}
\usepackage{pifont}
\usepackage[dvipdfmx]{graphicx}
\usepackage{float}
\newenvironment{claim}[1]{\par\noindent\underline{Claim:}\space#1}{}
\newenvironment{claimproof}[1]{\par\noindent\underline{Proof:}\space#1}{\hfill $\square$}
\newcommand{\R}{\mathbb{R}}
\newcommand{\Rp}{\mathbb{R}_{>0}}
\newcommand{\N}{\mathbb{N}}
\DeclareMathOperator{\spn}{\mathbb{Q}-span}
\begin{document}
\pagenumbering{arabic}
\title{第13回演習課題解答}
\author{習近平}
\maketitle
\newpage
\tableofcontents
\newpage
\setcounter{chapter}{13}
\section{問題13.1}
数列$(a_n)_{\N_{\ge 1}}$を次のように定める:
$$
a_n = \frac{1}{n^2}
$$
次に,級数$\sum\limits_{n=1}^{\infty} a_n$が収束することを以下示す:
数列$S_n = \sum\limits_{m=1}^{n} a_m$とする.数列$S_n$が単調増加である.次に,$S_n$が上に有界であることを示す:
$$
\sum_{m=1}^n  \frac{1}{m^2} \le \sum_{m=1}^n \frac{1}{m-1}- \frac{1}{m} \le 3 
$$
以上より,実数の連続性より$S_n$が収束する.続いて,$n \in \N_{\ge 1}$を任意に取った時,
$$
\sup_{x \in \R}  \frac{1}{x^2+n^2} \le \frac{1}{n^2} = a_n
$$
が成り立つ.M-testより,この函数項級数が$\R$上一様収束する.
\section{問題13.2}
区間$I=[-1,1]$とする.
\subsection{(1)}
$x\in \R\{0\}$を任意に取る.級数$\sum\limits_{n=1}^{\infty} \frac{x^2}{(1+x^2)^{n-1}}$は初項が$x^2$,公比が$\frac{1}{1+x^2}<1$の等比数列の無限和である.\\
この時,$\sum\limits_{n=1}^{\infty} \frac{x^2}{(1+x^2)^{n-1)}}=1+x^2$となる.\\
$x =0$の時,$\sum\limits_{n=1}^{\infty} 0 =0$となる.
$$
\sum_{n=1}^{\infty} \frac{x^2}{(1+x^2)^{n-1}} = \begin{cases}
1+ x^2, x\neq 0 \\
0,x=0
\end{cases}
$$
\subsection{(2)}
一様収束しない.\\
一様収束と仮定してこのもとで矛盾を導く:
第十三回講義の補題より,函数項級数を$f_n(x) =\sum\limits_{k=1}^{n} \frac{x^2}{(1+x^2)^{k-1}}$として,その一様収束先は(1)を用いて,次のような函数$f: \R \to \R$で表せる:
$$
f(x)  = \begin{cases}
1+ x^2, x\neq 0 \\
0,x=0
\end{cases}
$$
ここで各自然数$n$に対し,$f_n$が$I$上連続であり(有限個連続函数の和となるため,連続函数の性質に従う),また第十一回レポートの問題11.2の定理より,$f$は$I$上連続となる.しかし,$x=0$に対し,$\varepsilon =1/2$と定めたことろ,任意の$\delta \in \Rp$に関して,$x_0=\delta/2$としたら,$|x_0 -0|<\delta$を満たすが,$|f(x_0)-f(0)|>1>\varepsilon$で連続でないことを示せた.従って,矛盾が生じる.以上より一様収束するわけがありません.
\newpage
\section{\# 雑談}
最初は定義どおりに一様収束でないことを確認したかったが,値の評価が細かすぎてうまく行かなかった.過去のレポート問題を復習したら,この方針に辿り着いた.てか定義どおりに確認できるなら,ぜひ教えてください.
\end{document}

