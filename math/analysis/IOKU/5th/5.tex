\documentclass{jarticle}
\usepackage{xcolor}
\usepackage{amsfonts}
\usepackage{amssymb}
\usepackage{amsthm}
\usepackage{amsmath}
\begin{document}
\title{第5回演習課題}
\author{習近平}
\maketitle

\section*{問題5.1}
\noindent
(1)
\begin{proof}
$\lim\limits_{n \to \infty} \frac{a_{n+1}}{a_n}=l<1$を仮定して,$\sum\limits_{n=1}^{\infty}a_n$が収束することを導く:\\
$s=(l+1)/2, \varepsilon =s-l$とする.\\
この時,仮定より$\exists N \in \mathbb{N}; \forall n \in \mathbb{N} ; \left( n \ge N \rightarrow \left|\frac{a_{n+1}}{a_n} -l \right| < \varepsilon \right) $\\
$\forall n \in \mathbb{N} ; a_n >0$が成り立つため,以下の式が成り立つ:\\
\begin{equation}
\begin{aligned}
\frac{a_{n+1}}{a_n} &< \varepsilon +l \\
a_{n+1} &<(\varepsilon +l) a_n < sa_n
\end{aligned}
\end{equation}
次に数列$(b_n)_{n \in \mathbb{N} \setminus \{0\}}$を以下のように定める:\\
\begin{equation}
b_n := 
		\left\{ 
		\begin{aligned}
		&s^{n-N} a_N & n \ge N\\
		&max\{a_k|1 \le k \le N \} & n<N 
		\end{aligned} \right.
\end{equation}
この時,任意の$n \in \mathbb{N} \setminus \{0\}$に対し,\\
\begin{equation}
\begin{aligned}
\sum\limits_{k=1}^{n} b_k &=& \frac{1-s^{n-N}}{1-s} a_N + N \cdot max\{a_k|1 \le k \le N \} \\
			   &<& \frac{1}{1-s} a_N + N \cdot max\{a_k|1 \le k \le N \}
\end{aligned}
\end{equation}			   
が得られ,級数$\sum\limits_{n=1}^{\infty} b_n$が収束することがわかる,$\forall n \in  \mathbb{N} \setminus \{0\} ; a_n < b_n$という明らかな事実を用いて,
比較判定法より,級数$\sum\limits_{n=1}^{\infty} a_n$が収束することが示される.
\end{proof}
\newpage
\noindent
(2)\footnote{全く同じような議論する時はコードのコピペをすればうまく行く,楽勝w}
\begin{proof}
$\lim\limits_{n \to \infty} \frac{a_{n+1}}{a_n}=l>1$を仮定して,$\sum\limits_{n=1}^{\infty}a_n$が発散するすることを導く:\\
$s=(l+1)/2, \varepsilon =l-s$とする.\\
この時,仮定より$\exists N \in \mathbb{N}; \forall n \in \mathbb{N} ; \left( n \ge N \rightarrow \left|\frac{a_{n+1}}{a_n} -l \right| < \varepsilon \right) $\\
$\forall n \in \mathbb{N} ; a_n >0$が成り立つため,以下の式が成り立つ:\\
\begin{equation}
\begin{aligned}
\frac{a_{n+1}}{a_n} &> l - \varepsilon \\
a_{n+1} &>(l - \varepsilon) a_n > sa_n
\end{aligned}
\end{equation}
次に数列$(b_n)_{n \in \mathbb{N} \setminus \{0\}}$を以下のように定める:\\
\begin{equation}
b_n := 
		\left\{ 
		\begin{aligned}
		&s^{n-N} a_N & n \ge N\\
		&min\{a_k|1 \le k \le N \} & n<N 
		\end{aligned} \right.
\end{equation}
この時,$\alpha \in \mathbb{R}$を任意に取る:
任意の$n>N$を満たすような$n \in \mathbb{N} \setminus \{0\}$に対し,\\
\begin{equation}
\begin{aligned}
\sum\limits_{k=1}^{n} b_k &=& \frac{1-s^{n-N}}{1-s} a_N + N \cdot min\{a_k|1 \le k \le N \} \\
			   &>& \frac{s(n-N)}{s-1} a_N + N \cdot max\{a_k|1 \le k \le N \}
\end{aligned}
\end{equation}
アルキメデス性よりある$N' \in \mathbb{N}$が存在し,$n>N'$を満たす任意の$n \in \mathbb{N}$に対し$\sum\limits_{k=1}^{n} b_k > \alpha$となる.\\			   
級数$\sum\limits_{n=1}^{\infty} b_n$が発散することがわかる,$\forall n \in  \mathbb{N} \setminus \{0\} ; a_n > b_n$という明らかな事実を用いて,
比較判定法より,級数$\sum\limits_{n=1}^{\infty} a_n$も発散することが示された.
\end{proof}
\newpage
\section*{問題5.2}
\noindent
上の問題5.1の結論を用いて証明をごまかしていく:\\
(1)収束する:\\
\begin{proof}
$a_n = \frac{n^2}{2^n}$とする,この時$\lim\limits_{n \to \infty} \frac{a_{n+1}}{a_n}=1/2$を以下示す:\\
$\varepsilon \in \mathbb{R}_{>0} $を任意に取る,この時,アルキメデス性より,$2/(N'-1) < \varepsilon$を満たす$N' \in \mathbb{N}$が存在する,これを一つ取る.\\
このもとで,$N := max{2,N'}$と定め,$n \ge N $を満たす任意の$n \in \mathbb{N}$に対し,以下の式が成り立つ:\\
\begin{equation}
\begin{aligned}
\left| \frac{a_{n+1}}{a_n} - \frac{1}{2} \right| = \left| \frac{1}{2} \left( 1+\frac{1}{n} \right)^2 - \frac{1}{2} \right| < \frac{2}{n} < \varepsilon
\end{aligned}
\end{equation}
従って,$\lim\limits_{n \to \infty} \frac{a_{n+1}}{a_n}=1/2$が示され,問題5.1より,$\sum\limits_{n=1}^{\infty} \frac{n^2}{2^n} $が収束することが示された.
\end{proof}
(2)発散する:\\
\begin{proof}
	$a_n = \frac{(2n)!}{(n!)^2}$と定め,$\lim\limits_{n \to \infty} \frac{a_{n+1}}{a_n}=4$であることを以下示す:\\
	$\varepsilon \in \mathbb{R}_{>0}$を任意に取る.アルキメデス性より$\frac{2}{N'} < \varepsilon$を満たす$N' \in \mathbb{N}$が存在する.\\
	$N= max \{2,N'+1\} $と定め,$n \ge N$を満たす任意の$n \in \mathbb{N}$に対し,以下の式が成り立つ:\\
	\begin{equation}
		\begin{aligned}
			&	\left|	\frac{a_{n+1}}{a_n}-4 \right|& = & \left| \frac{(2n+1)! (n!)^2}{(2n)!((n+1)!)^2} -4 \right| \\
			&				     & = & \frac{2}{n+1}< \frac{2}{N} <\varepsilon
		\end{aligned}
	\end{equation}
	従って,$\lim\limits_{n \to \infty} \frac{a_{n+1}}{a_n}=4>1$が示され,問題5.1より,級数$\sum\limits_{n=1}^{\infty} \frac{(2n)!}{(n!)^2}$が発散することが示された.
\end{proof}
(3)収束する:\\
\begin{proof}
	$a_n=\frac{2^n}{n^n}$と定め,$\lim\limits_{n \to \infty} \frac{a_{n+1}}{a_n}=0$であることを以下示す:\\
	まず,講義中に扱っていない$\lim\limits_{n \to \infty} (1+\frac{1}{n})^n = e \in \mathbb{R}_{>0}$であることを証明せずに使わせていただく.\footnote{正直これは示せるが,コードを打つのがしんどいし,早く講義中に指数対数を定義してよ,毎回こんな脚注を書いて,これをみるのも面倒くさいでしょう?}\\
	$\varepsilon \in \mathbb{R}_{>0}$を任意に取る,この時,先ほど宣言したことと,収束数列の四則演算に関する性質より,ある自然数$N'$が存在し,$n' \ge N'$を満たす任意の自然数$n'$に対し,次のようなことがわかる:\\
	$$
	\left|  \frac{n'^{n'}}{(n'+1)^{n'}} - \frac{1}{e} \right| < \varepsilon
	$$
	また実数のアルキメデス性より,ある自然数$N''$が存在し,$\frac{2(\frac{1}{e}+ \varepsilon)}{N''+1}< \varepsilon$を満たす.\\
	この時$N= max\{ N', N'' \}$とする,$n \ge N$を満たす任意の自然数$n$に対し,\\
	\begin{equation}
		\begin{aligned}
			&\left| \frac{a_{n+1}}{a_n} -0 \right| &=& \frac{2^{n+1}n^n}{(n+1)^{n+1} 2^n} =  \frac{2 \cdot n^n}{(n+1)^{n}(n+1)}\\
			&				      &\le& \frac{2(\frac{1}{e} +\varepsilon)}{n+1} <\varepsilon&
		\end{aligned}
	\end{equation}
	以上の議論に踏まえ,問題5.1の結論を施すと級数$\sum\limits_{n=1}^{\infty} \frac{2^n}{n^n}$が収束することが示された.
\end{proof}
\section*{問題5.6}
まず収束に関する部分を以下示す:
\begin{proof}
	$\gamma = (\alpha +1)/2$として,$\varepsilon = \alpha - \gamma$とする.この時$\lim\limits_{n \to \infty} n \left( \frac{a_n}{a_{n+1}} -1 \right) = \alpha$であるため.\\
	ある$N \in \mathbb{N}$が存在し,$ n \ge N$を満たす任意の$ n \in \mathbb{N}$に対し,次のような関係が成り立つ:\\
			$$
			\left| n \left(\frac{a_n}{a_{n+1}}-1 \right) - \alpha \right|  <  \varepsilon 
			$$

			$$
			n \left(\frac{a_n}{a_{n+1}}-1 \right)  >  \gamma
			$$

			$$
			na_n-na_{n+1} > \gamma a_{n+1} 
			$$
			ここで,左辺を階差数列の形に直すことを目的にして,変形を施すと次にようになる:\\
			$$
			na_n - (n+1)a_{n+1}  >  ( \gamma -1 ) a_{n+1}
			$$
			この式に対し,$k=N$から$k=n$まで動かして両辺足していくと次のようになる:\\
			$$
			(\gamma -1) \sum_{k=N}^{n}a_{k+1} <Na_N - (n+1)a_{n+1}<Na_N
			$$
			変形して有限項を足せば以下の式になる:\\
			$$
			\sum_{k=1}^{n}a_k < \frac{Na_N}{\gamma -1} + \sum_{k=1}^{N-1}a_k
			$$
	
	以上より正項級数$\sum\limits_{n=1}^{\infty}a_n$が上に有界であることがわかり,その収束性が示された.\\
\end{proof}
次に発散に関する部分を示す:
\begin{proof}
第三回レポート問題で積分法を用いて調和級数を評価したため,本回答では調和級数が発散することを証明せずに用いるとする:
	$\delta = (\alpha +1)/2$として,$\varepsilon = \delta - \alpha$とする.この時$\lim\limits_{n \to \infty} n \left( \frac{a_n}{a_{n+1}} -1 \right) = \alpha$であるため.\\
        ある$N \in \mathbb{N}$が存在し,$ n \ge N$を満たす任意の$ n \in \mathbb{N}$に対し,次のような関係が成り立つ:\\
			$$
			\left| n \left(\frac{a_n}{a_{n+1}}-1 \right) - \alpha \right|  <  \varepsilon 
			$$
			$$
			n \left(\frac{a_n}{a_{n+1}}-1 \right) <  \delta 
			$$
			$$
			na_n - n a_{n+1}  <  \delta a_{n+1} 
			$$
			$$
			na_n<(n+\delta)a_{n+1}<(n+1)a_{n+1}
			$$
			ここで数列$(na_n)$が$n \ge N$の時狭義単調増加で$n \ge N$を満たす時の下限$I$を取る, $I > 0$が正項級数であることより明らか
			$$
			a_n > \frac{I}{n}
			$$
			$$
			\sum_{k=N}^{n} a_k > I \sum_{k=N}^{n}\frac{1}{k}
			$$
			$$
			\sum_{n=1}^{\infty} a_n > \sum_{n=1}^{N-1}a_n + I \sum_{n=N}^{\infty} \frac{1}{n}
			$$
	右辺が調和級数であるため,発散する,比較判定法より正項級数$\sum\limits_{n=1}^{\infty}a_n$が発散することが示された.
\end{proof}
ダランベールの判定法より判定できず,ラーベの判定法がうまく行くのが講義中に扱っていた$\sum_{n=1}^{\infty} \frac{1}{n^2}$である.$a_n = \frac{1}{n^2}$と置いたところ,
$\lim_{n \to \infty}\frac{a_{n+1}}{a_{n}}=1; \lim_{n \to \infty}n \left(\frac{a_n}{a_{n+1}}-1 \right) =2>1$,実際に収束することが講義中で説明されたためここで割愛する.
\\
\section*{\# 雑談}
はい,ではこれから例の雑談に入ろう.今回の問題5.2(3)で重要な極限自然対数の底の逆数が出たのではないか?あとラーベの判定法で調和級数が出てきて,この判定法は微分法と何かの関連性でも持っているの?










\end{document} 
