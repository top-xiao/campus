\documentclass{jreport}
\usepackage{xcolor}
\usepackage{amsmath}
\usepackage{amsfonts}
\usepackage{amssymb}
\usepackage{amsthm}
\usepackage{bm}
\usepackage{romannum}
\usepackage[dvipdfmx,hidelinks]{hyperref}
\usepackage{pxjahyper}
\usepackage{framed}
\usepackage{pifont}
\usepackage[dvipdfmx]{graphicx}
\usepackage{float}
\newenvironment{claim}[1]{\par\noindent\underline{Claim:}\space#1}{}
\newenvironment{claimproof}[1]{\par\noindent\underline{Proof:}\space#1}{\hfill $\square$}
\newcommand{\R}{\mathbb{R}}
\newcommand{\Rp}{\mathbb{R}_{>0}}
\newcommand{\N}{\mathbb{N}}
\DeclareMathOperator{\spn}{\mathbb{Q}-span}
\begin{document}
\pagenumbering{arabic}
\title{第12回演習課題解答}
\author{習近平}
\maketitle
\newpage
\tableofcontents
\newpage
\setcounter{chapter}{12}
\section{問題12.1}
\subsection{(1)}
函数列$(f_n)$は区間$I$で$f$に一様収束し,$f$は$I$で連続とする.この時,区間$I$上の数列$(x_n)$は点$a \in I$に収束するとする.以下の論理式が成立することを示す:
$$
\forall \varepsilon \in \Rp; \exists N \in \N; \forall n \in \N;\left( n\ge N \implies \left|f_n(x_n)-f(a) \right| < \varepsilon \right)
$$
まず,$\varepsilon \in \Rp$を任意に取る,$(f_n)$は区間$I$で$f$に一様収束することより,
$$
\exists N' \in \N;\forall x \in I; \forall n \in \N;\left( n\ge N' \implies \left| f_n(x) - f(x) \right| <\frac{\varepsilon}{2} \right)
$$
が成立する.このような自然数$N'$を一つ取る.次に,$f$が区間$I$で連続することより,
$$
\exists \delta \in \Rp; \forall x \in I; \left( |x-a|<\delta \implies |f(x) -f(a)|<\frac{\varepsilon}{2} \right)
$$
も成立するから,このような$\delta \in \Rp$を一つ取る.また,数列$(x_n)$は 点$a $に収束する数列であるため,
$$
\exists N'' \in \N; \forall n \in \N; \left( n\ge N'' \implies |x_n - a|<\delta \right)
$$
が得られる.この場合,このような自然数$N''$を一つ取り,$N = \max\{ N',N''\}$と定め,自然数$n$を$n\ge N$を満たすように任意に取る,
\begin{equation}
\begin{aligned}
	|f_n(x_n) -f(a)|&=|f_n(x_n) - f(x_n) +f(x_n) -f(a) |\\
			& \le |f_n(x_n) -f(x_n)| + |f(x_n) - f(a)| \\
			&<\frac{\varepsilon}{2} +\frac{\varepsilon}{2} =\varepsilon
\end{aligned}
\end{equation}
\newpage
\subsection{(2)}
成り立たない.
$$
f_n(x)=\begin{cases}
	x^n , x\in (0,1)\\
	0,x=1
	\end{cases}
$$
この函数列は$f(x)=0$に各点収束する,また数列$x_n=1-1/n$に対し,$\lim_{n\to \infty}x_n =1$が成り立つ.\\
$f(1)=0$となる,しかし$f_n(x_n)=(1-1/n)^{n}$で,$\lim_{n\to\infty}f_n(x_n)=1/e$.よって,矛盾が生じた\\
\newpage
\section{\# 雑談}
(2)は(1)の証明を振り返ることで簡単にわかるだろう.今週もネタ尽きでここまでにしておきます.お疲れ様でした.
\end{document}
