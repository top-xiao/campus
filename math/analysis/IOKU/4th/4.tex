\documentclass{jarticle}
\usepackage{bm}
\usepackage{xcolor}
\usepackage{amsmath}
\usepackage{amsfonts}
\usepackage{amssymb}
\usepackage{amsthm}
\begin{document}
\title{第4回レポート問題}
\author{習近平}
\maketitle
\section*{Q4.1}
\begin{proof}
$\sum\limits_{n=1}^{\infty} \mid a_{n} \mid$が収束すると仮定して,$\sum\limits_{n=1}^{\infty} a_{n} $が収束することを導く:\\
$\mathbb{R}$の連続性より,数列$ \mathbf{\widetilde{S}}_{n} = \sum\limits_{k=1}^{n} \mid a_{k} \mid$がCauchy列であることのもとで,\\
数列$\mathbf{S}_{n} = \sum\limits_{k=1}^{n} a_{k} $もCauchy列であることを以下示す:\\
$\varepsilon \in \mathbb{R}_{>0} $を任意に取る.\\
仮定より,ある自然数$N \in \mathbb{N}$が存在し,このような$N$を一つとると以下の命題が成り立つ.\\
$m \ge n \ge N$を満たす任意の$m,n \in \mathbb{N}$に対し,次の式が成立する:\\
$$
\mid \mathbf{\widetilde{S}}_{m} - \mathbf{\widetilde{S}}_{n} \mid \; = \;  \bm{\mid} \sum_{k=n+1}^{m} \mid a_{k} \mid \bm{\mid}< \varepsilon
$$
この時,\\
$$
\mid \mathbf{S}_{m} - \mathbf{S}_{n} \mid =\, \mid \sum_{k=n+1}^{m}  a_{k}  \mid
$$
に三角不等式を帰納的に適用していくと,次の関係が得られる:\\
$$
\mid \mathbf{S}_{m} - \mathbf{S}_{n} \mid =\; \mid \sum_{k=n+1}^{m}  a_{k}  \mid\; < \;\bm{\mid} \sum_{k=n+1}^{m} \mid a_{k} \mid \bm{\mid} \; < \varepsilon
$$
従って,数列$(\mathbf{S_{n}})_{n \in \mathbb{N}\setminus }$がCauchy列であることがわかり,実数の連続性より,数列$(\mathbf{S_{n}})_{n \in \mathbb{N}\setminus }$が$\mathbb{R}$上の収束列であり,
$\sum\limits_{n=1}^{\infty} a_{n} $が収束することが示された.
\end{proof}

\section*{Q4.2}
(1)\\
\begin{proof}
\begin{equation}
	b_{n} = \left\{
		\begin{aligned}
			&1 &n=1& \\
			&\frac{1}{n(n-1)}& n>1&\\
		\end{aligned}
			\right\|
	\\
	\\
	\\
	a_{n}= \frac{n}{(n+1)^{3}}
\end{equation}
とする.この時,
$$
	0 \le a_n \le \frac{1}{n^2} \le b_n
$$
が成り立つため,$\sum\limits_{n=1}^{\infty} \frac{n}{(n+1)^3}$が正項級数であることがわかり,$\sum\limits_{n=1}^{\infty} \frac{n}{(n+1)^3}$が上に有界であることを以下示す:\\
$N \ge 2$を満たす任意の$N \in \mathbb{N}$に対し,
$$
	\sum_{n=1}^N b_n = 1+1-\frac{1}{N} =2-\frac{1}{N}
$$
従って,$(\sum\limits_{k=1}^n b_k)_{n \in \mathbb{N}}$が上に有界で,$(\sum\limits_{k=1}^n a_k) \le (\sum\limits_{k=1}^n b_k)$より,\\
$(\sum\limits_{k=1}^n a_k)_{n \in \mathbb{N}}$も上に有界,実数の連続性より,\\
$
	\lim\limits_{n \to \infty} \sum\limits_{k=1}^{n}a_{k}
$
	が存在する.以上より,級数$\sum\limits_{n=1}^{\infty} a_n$が収束することが示された.
\end{proof}
(2)\\
\begin{proof}
数列$(a_{n})_{n \in \mathbb{N}\setminus \{0\}}$ と数列$(b_{n})_{n \in \mathbb{N}\setminus \{0\}}$を以下のように定める:\\
\begin{equation}
 b_{n} = \left\{
                \begin{aligned}
                        &1 &n=1,2& \\
                        &\frac{6}{n(n-1)}& n>2&\\
                \end{aligned}
                        \right\|
        \\
        \\
        \\
        a_{n}= \frac{n}{3^n+1}
\end{equation}
$n=1,2$の時,以下の式が成り立つ:\\
\begin{equation}
	\begin{aligned}
		0 & < a_1 &=& \frac{1}{4} &<b_1&=1\\
		0 & < a_2 &=& \frac{1}{5} &<b_2&=1\\
	\end{aligned}
\end{equation}
$n>2$の時:\\
	\begin{equation}
		\begin{aligned}
			0 < a_n &=& \frac{n}{3^n+1} < \frac{n}{2^n} \\
				&<& \frac{n}{(1+1)^n}<\frac{6}{n^2}<\frac{6}{n(n-1)}=b_n\\
		\end{aligned}
	\end{equation}
以上より,$(\sum\limits_{k=1}^n a_k)_{n \in \mathbb{N} \setminus \{0\}}$が正項級数がわかり,その有界性を以下示す:
$N \ge 3$を満たす任意の自然数$N$に対し,次の関係が成り立つ:\\
$$
	\sum_{k=1}^Nb_k \le 6(1+1+ \frac{1}{2}-\frac{1}{N}) < 15
$$
なので,$(\sum\limits_{k=1}^n b_k)_{n \in \mathbb{N} \setminus \{0\}}$が上に有界で,よって$(\sum\limits_{k=1}^n a_k)_{n \in \mathbb{N} \setminus \{0\}}$も上に有界である.\\
以上より,級数$\sum\limits_{n=1}^{\infty} a_n$が収束することが示された.





\end{proof}

\section*{\#雑談}
今週の数学序論Bは重そうなので,おまけ問題の提出を遅らせていただきます.それらの性質を全て満たす面積函数が
ただ一つ存在することから類推して,以前やった行列式も性質から決められるでしょう.あと小学生に説明する問題に関して,こういうふうに定義したから,覚えなさいというしかないと感じている.(教職とっていないから日本社会が救われる世界線w)
























\end{document}
