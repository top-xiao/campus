\documentclass{jarticle}
\usepackage{amsmath}
\usepackage{amsfonts}
\usepackage{amssymb}
\usepackage{amsthm}
\usepackage{xcolor}
\usepackage{bm}
\usepackage{mathtools}

\begin{document}
\title{適当な課題の解答解説}
\author{習近平}
\date{$\forall \varepsilon \in \mathbb{R}_{>0} ; deadline - \varepsilon < \text{提出時刻} <deadline$}
\maketitle
\noindent
\fbox{1}\\
先に結論を言う:函数$f : \mathbb{R}^2 \mapsto \mathbb{R}$が原点において連続,偏微分可能,全微分可能である.\\
ここでこの函数の全微分可能性だけ示して,全微分可能性に第三回講義ノートの
$Thm(1),(2)$を適用すれば,残りの主張が正しいことが直ちに分かる.
\begin{proof}
$\bm{h} =(h,k) \in \mathbb{R}^2 , \bm{m} = (0,0) \in \mathbb{R}^2$とする.\\
$ \varepsilon \in \mathbb{R}_{>0} $を任意に取る.この時$ \delta = \frac{2 \varepsilon }{ \pi }$とする.\\
$ \mid \mid \bm{h} \mid \mid < \delta $を満たす任意の$ \bm{h} \in \mathbb{R}^2$に対し,\\
$$
\begin{aligned}
\left| \cfrac{f(\bm{h}+\bm{0})-f(\bm{0}) - < \bm{m} \cdot \bm{h}>}{ \mid \mid \bm{h} \mid \mid } \right| & = \left| \cfrac{f(\bm{h})}{\mid \mid \bm{h} \mid \mid} \right|  \le  \cfrac{\cfrac{\pi}{2} \mid h^2 \mid }{\mid \mid \bm{h} \mid \mid } \\								       
& \le  \frac{\pi}{2}\mid \mid \bm{h} \mid \mid	 <  \varepsilon
\end{aligned}
$$
よって,原点における全微分可能性が示された.第三回講義ノートの$Thm(1),(2)$より,原点における連続性と偏微分可能性も示された.
\end{proof}
次に,2つの二階偏導関数を求める:\\
$$
\left. \frac{\partial f}{\partial x \partial y}\right|_{(x,y)=(0,0)}=1
$$
$$
\left. \frac{\partial f}{\partial y \partial x}\right|_{(x,y)=(0,0)}=0
$$
以上の式は偏導関数を求める時に1階導関数を普通に微分して,二階導関数を定義に従ってやれば以上になることが明らかである.\\
\\
\noindent \fbox{2}
\begin{proof}
$\mathbb{R}^2 $上の二項関係$\sim$を次のように定める:\\
$$
(x,y) \sim (x',y') \Longleftrightarrow ax+by=ax'+by'
$$
$\sim$が$\mathbb{R}^2$上の同値関係であることが自明である,念の為一応示しておく:\\
\begin{tabular}{|p{10cm}|}
\hline
反射律:\\
$(x,y) \in \mathbb{R}^2$を任意に取る. $ax+by=ax+by$が成り立つため,$(x,y) \sim (x,y)$が成り立つ.\\
\hline
対称律:\\
$(x_1,y_1) \sim (x_2,y_2)$を満たすような$(x_1,y_1),(x_2,y_2) \in \mathbb{R}^2$を任意に取る.\\
$(x_1,y_1) \sim (x_2,y_2)$が成り立つため.\\
$ax_1+by_1=ax_2+by_2$が得られ,$ax_2+by_2=ax_1+by_1$が成り立つ.よって,$(x_2,y_2) \sim (x_1,y_1)$\\
\hline
推移律:\\
$(x_1,y_1) \sim (x_2,y_2)$かつ$(x_2,y_2) \sim (x_3,y_3)$が成り立つような$(x_1,y_1),(x_2,y_2),(x_3,y_3) \in \mathbb{R}^2$を任意に取る.\\
この時,$ax_1+by_1=ax_2+by_2=ax_3+by_3$が成り立つため,$(x_1,y_1) \sim (x_3,y_3)$が示された.\\
\hline
\end{tabular}
\\
\noindent 以上の議論より$\sim$が$\mathbb{R}^2$上の同値関係であることが示された.\\
$\mathbb{R}^2$を$\sim$で割った商集合を$\mathbb{R}^2 / \sim$とする.この時,次のような商写像はwell-definedであることが構成より明らかである\footnote{ここまで丁寧に同値関係であることを議論したのに,わからなかったら人生ガチャをやり直そう,まだ間に合うよw}:
\begin{align}
 h: \mathbb{R}^2 / \sim \mapsto \mathbb{R}\\
 h([(x,y)]) := ax+by
\end{align}
しかもこの写像が全単射である\footnote{任意の$y \in \mathbb{R}$に対し,$[(0,y/b)]$を取れは全射であることが明らか,単射性は構成より明らかである}.
次の写像$s : \mathbb{R}^2 / \sim \mapsto \mathbb{R}$がwell-definedであることを以下示す:\\
$$
s([(x,y)]) := f(x,y)
$$
$(x_1,y_1) \sim (x_2,y_2)$を満たすような$\bm{x_{1}}=(x_1,y_1),\bm{x_{2}}=(x_2,y_2) \in \mathbb{R}^2$を任意に取る.\\
$\Delta \bm{x}= \bm{x_2}-\bm{x_1}$とする.\\ 
$$
b \frac{\partial f}{\partial x} \bm{x} = a \frac{\partial f}{\partial y} \bm{x}
$$
より,$f(\bm{x}+k(a,-b))=f(\bm{x}) \quad ;\quad \left(k \in \mathbb{R} \right)$.\\
この時,ある実数$k$が存在し$\Delta \bm{x} = k(a,-b)$を満たすため .$s([(x_1,y_1)]) = s([(x_2,y_2)])$が得られ,\\
従って,写像sがwell-definedである.\\
以上の議論より写像$(s \circ h^{-1}) : \mathbb{R} \mapsto \mathbb{R}$が問題文を満たす写像$g$である.\\
\end{proof}
\\
\noindent 
\fbox{3}
\begin{proof}
$C^2$級函数$f$について\\
$$
\frac{\partial^2 f}{\partial x^2} = \frac{\partial^2 f}{\partial x \partial y}=\frac{\partial^2 f}{\partial y^2}=0
$$
が成り立つため.ティラーの定理を適用すると次のようになる.\\
\begin{equation}
\begin{aligned}
&\exists \theta \in (0,1);\\
&f(\bm{x})&=&f(\bm{0})+(x\frac{\partial}{\partial x}+y\frac{\partial}{\partial y})f(\bm{0})+ \frac{1}{2}(x\frac{\partial}{\partial x}+y\frac{\partial}{\partial y})^2 f(\theta \bm{x})\\
&&=&f(\bm{0})+(x\frac{\partial}{\partial x}+y\frac{\partial}{\partial y})f(\bm{0})\\
&&=&f(\bm{0} ) + \frac{\partial f(\bm{0})}{\partial x}\cdot x+ \frac{\partial f(\bm{0})}{\partial y} \cdot y
\end{aligned}
\end{equation}
$\frac{\partial f(\bm{0})}{\partial x}, \frac{\partial f(\bm{0})}{\partial y}$が定数であるため.
$\frac{\partial f(\bm{0})}{\partial x}=a, \frac{\partial f(\bm{0})}{\partial y}=b,f(\bm{0})=c$とすれば,
$f(x,y)=ax+by+c$が成り立ち,問題の主張が示された.
\end{proof}



\end{document}

