\documentclass[dvipdfmx]{jarticle}
\usepackage[dvipdfmx]{graphicx}
\usepackage{float}
\usepackage{framed}
\usepackage{pifont}
\usepackage[dvipdfmx,hidelinks]{hyperref}
\usepackage{pxjahyper}
\usepackage{bm}
\usepackage{tikz}
\usepackage{amsmath}
\usepackage{amsfonts}
\usepackage{amssymb}
\usepackage{amsthm}
\usepackage[all]{xy}
\usepackage[english]{babel}
\usepackage{import}
\usepackage{xifthen}
\usepackage{pdfpages}
\usepackage{transparent}
\newcommand{\cl}[1]{\overline{ #1}  }
\newcommand{\Int}[1]{#1 ^{\mathrm{o}} }
\newcommand{\bd}[1]{\operatorname{Bd}{#1}}
\newtheorem{theorem}{Theorem}[section]
\newtheorem{corollary}{Corollary}[theorem]
\newtheorem{lemma}[theorem]{Lemma}
\newtheorem{definition}{Definition}[section]
\newtheorem{claim}{Claim}[section]
\theoremstyle{remark}
\newtheorem{remark}{Remark}[section]
\theoremstyle{plain}
\newtheorem{example}{Example}[section]
\newtheorem{exercise}{Exercise}[section]
\newcommand{\incfig}[1]{%
    \includegraphics{~/Documents/campus/math/eps/topology/1st/#1.eps}
}

\begin{document}
\title{平面におけるベクトル解析の基本常識}
\author{習近平}
\date{\today}
\maketitle
$$
\int_{\it{l}} V \cdot dl = \int_a^b V(l(t)) \cdot \frac{dl}{dt} \: dt
$$
$$
\int_{\it{L}}V \cdot \vec{n} dL = \int_{a}^{b} {\| \dot{l}(s) \| V(l(s)) \vec{n}  } \: d{s}
$$
$$
\int_{\it{L}} V \cdot \vec{n} \: dL = \iint_D \mathrm{div}{V} \: dxdy
$$
$$
\iint_D \mathrm{rot} V \: dxdy =\iint_D \left( \frac{\partial V_2}{\partial x} - \frac{\partial V_1}{\partial y} \right) \: dxdy = \int_{\partial D} V \cdot dl
$$
三次元ベクトルの外積が可換性を持たない。\\
スカラー三重積は行列式で表される。\\
座標変換が行われた場合$\rm{div}, \rm{rot}$ を合成函数の微分を用いて変換しなければならない.








\end{document}
